% Created 2013-03-11 lun 13:58
\documentclass[xcolor={usenames,svgnames,dvipsnames}]{beamer}
\usepackage[utf8]{inputenc}
\usepackage[T1]{fontenc}
\usepackage{fixltx2e}
\usepackage{graphicx}
\usepackage{longtable}
\usepackage{float}
\usepackage{wrapfig}
\usepackage{soul}
\usepackage{textcomp}
\usepackage{marvosym}
\usepackage{wasysym}
\usepackage{latexsym}
\usepackage{amssymb}
\usepackage{hyperref}
\tolerance=1000
\usepackage{color}
\usepackage{listings}
\AtBeginSection[]{\begin{frame}[plain]\tableofcontents[currentsection,hideallsubsections]\end{frame}}
\lstset{keywordstyle=\color{blue}, commentstyle=\color{gray!90}, basicstyle=\ttfamily\small, columns=fullflexible, breaklines=true,linewidth=\textwidth, backgroundcolor=\color{gray!23}, basewidth={0.5em,0.4em}, literate={á}{{\'a}}1 {ñ}{{\~n}}1 {é}{{\'e}}1 {ó}{{\'o}}1 {º}{{\textordmasculine}}1}
\usepackage{mathpazo}
\usefonttheme{serif}
\usecolortheme{rose}  
\usetheme{Goettingen}
\hypersetup{colorlinks=true, linkcolor=Blue, urlcolor=Blue}
\usepackage{fancyvrb}
\DefineVerbatimEnvironment{verbatim}{Verbatim}{fontsize=\tiny, formatcom = {\color{black!70}}}
\providecommand{\alert}[1]{\textbf{#1}}

\title{Crear paquetes en R}
\author{Oscar Perpiñán Lamigueiro}
\date{Marzo de 2013}
\hypersetup{
  pdfkeywords={},
  pdfsubject={},
  pdfcreator={Emacs Org-mode version 7.8.11}}

\begin{document}

\maketitle


\section{Introducción}
\label{sec-1}
\begin{frame}
\frametitle{Definiciones previas}
\framesubtitle{\href{http://cran.r-project.org/doc/contrib/Leisch-CreatingPackages.pdf}{Creating R Packages: a tutorial}}
\label{sec-1-1}

\begin{itemize}
\item \textbf{Package}: An extension of the R base system with code, data and
  documentation in standardized format.
\item \textbf{Library}: A directory containing installed packages.
\item \textbf{Repository}: A website providing packages for installation.
\item \textbf{Source}: The original version of a package with human-readable text and code.
\item \textbf{Binary}: A compiled version of a package with computer-readable
  text and code, may work only on a specific platform.
\item \textbf{Base packages}: Part of the R source tree, maintained by R Core.
\item \textbf{Recommended packages}: Part of every R installation, but not
  necessarily maintained by R Core.
\item \textbf{Contributed packages}: All the rest.
\end{itemize}
\end{frame}
\begin{frame}
\frametitle{¿Por qué crear y publicar un paquete para \texttt{R}?}
\label{sec-1-2}


\begin{itemize}
\item Es una herramienta cómoda para mantener colecciones coherentes de funciones
  y datos.
\item Permite publicar código de forma que pueda ser empleado por
  otros siguiendo unas estructuras comunes.
\item La estructura de un paquete obliga a organizar, limpiar y
  documentar el código.
\item Al distribuir las herramientas para que otros puedan usarlas, se
  obtiene realimentación sobre lo publicado, de forma que se
  aumenta su robustez, se amplian sus funcionalidades y se conecta
  con otras herramientas y proyectos.
\item ``As we enjoy great advantages from the inventions of others, we
  should be glad of an opportunity to serve others by any
  invention of ours, and this we should do freely and generously.''
\end{itemize}
\end{frame}
\begin{frame}
\frametitle{Algunos consejos genéricos}
\label{sec-1-3}


Extraído de \href{http://arxiv.org/abs/1210.0530}{Best Practices for Scientific Computing}

\begin{itemize}
\item Write programs for people, not computers.
\item Automate repetitive tasks
\item Use the computer to record history
\item Make incremental changes
\item Use version control
\item Don't repeat yourself (or others)
\item Plan for mistakes
\item Optimize software only after it works correctly
\item Document design and purpose, not mechanics
\item Collaborate
\end{itemize}
\end{frame}
\begin{frame}
\begin{block}{Referencias}
\label{sec-1-4-1}

\begin{itemize}
\item \href{http://cran.r-project.org/doc/manuals/r-release/R-exts.html}{Writing R Extensions}
\item \href{http://cran.r-project.org/doc/contrib/Leisch-CreatingPackages.pdf}{Creating R Packages: a tutorial}
\end{itemize}
\end{block}
\end{frame}
\section{Crear un paquete en R}
\label{sec-2}
\begin{frame}
\frametitle{Estructura}
\label{sec-2-1}

\begin{itemize}
\item Las fuentes de un paquete de \texttt{R} están contenidas en un
  directorio que contiene al menos:
\begin{itemize}
\item Los ficheros DESCRIPTION y NAMESPACE
\item Los subdirectorios:
\begin{itemize}
\item \texttt{R}: código en ficheros .R
\item \texttt{man}: páginas de ayuda de las funciones, métodos y clases
      contenidos en el paquete.
\end{itemize}
\end{itemize}
\item Esta escructura puede ser generada con \texttt{package.skeleton}.
\end{itemize}
\end{frame}
\begin{frame}[fragile]
\frametitle{DESCRIPTION}
\label{sec-2-2}

\begin{itemize}
\item El fichero DESCRIPTION contiene la información básica sobre el
  paquete con un formato preestablecido:
\end{itemize}

\begin{verbatim}
Package: pkgname
Version: 0.5-1
Date: 2004-01-01
Title: My First Collection of Functions
Authors@R: c(person("Joe", "Developer", role = c("aut", "cre"),
                     email = "Joe.Developer@some.domain.net"),
              person("Pat", "Developer", role = "aut"),
              person("A.", "User", role = "ctb",
                email = "A.User@whereever.net"))
Author: Joe Developer and Pat Developer, with contributions from A. User
Maintainer: Joe Developer <Joe.Developer@some.domain.net>
Depends: R (>= 1.8.0), nlme
Suggests: MASS
Description: A short (one paragraph) description of what
  the package does and why it may be useful.
License: GPL (>= 2)
URL: http://www.r-project.org, http://www.another.url
\end{verbatim}
\begin{itemize}
\item Los campos \texttt{Package}, \texttt{Version}, \texttt{License}, \texttt{Title}, \texttt{Author} y
  \texttt{Maintainer} son obligatorios.
\end{itemize}
\end{frame}
\begin{frame}
\frametitle{NAMESPACE}
\label{sec-2-3}

\begin{itemize}
\item \texttt{R} usa un sistema de gestión de \emph{espacio de nombres} que
  permite al autor del paquete especificar:
\begin{itemize}
\item las variables del paquete que se exportan (y son, por tanto,
    accesibles a los usuarios)
\item las variables que se importan de otros paquetes.
\item las clases y métodos \texttt{S3} y \texttt{S4} que deben registrarse.
\end{itemize}
\item Este mecanismo queda definido en el contenido del fichero
  \texttt{NAMESPACE}.
\end{itemize}
\end{frame}
\begin{frame}
\frametitle{NAMESPACE}
\label{sec-2-4}

\begin{itemize}
\item El \texttt{NAMESPACE} controla la estrategia de busqueda de variables
  que utilizan las funciones del paquete:
\begin{itemize}
\item En primer lugar busca entre las creadas localmente (por el código de la carpeta \texttt{R/}).
\item En segundo lugar busca entre las variables importadas
    explicitamente de otros paquetes.
\item En tercer lugar busca en el \texttt{NAMESPACE} del paquete \texttt{base}.
\item Por último busca siguiendo el camino habitual (ver el
    resultado de \texttt{search()})
\end{itemize}
\end{itemize}
\end{frame}
\begin{frame}[fragile]
\frametitle{NAMESPACE}
\label{sec-2-5}

\begin{itemize}
\item Para exportar las variables \texttt{f} y \texttt{g}:
\end{itemize}

\lstset{language=R}
\begin{lstlisting}
export(f, g)
\end{lstlisting}
\begin{itemize}
\item Para importar \textbf{todas} las variables del paquete \texttt{pkgExt}:
\end{itemize}

\lstset{language=R}
\begin{lstlisting}
import(pkgExt)
\end{lstlisting}
\begin{itemize}
\item Para importar las variables \texttt{var1} y \texttt{var2} del paquete
  \texttt{pkgExt}:
\end{itemize}

\lstset{language=R}
\begin{lstlisting}
importFrom(pkgExt, var1, var2)
\end{lstlisting}
\end{frame}
\begin{frame}[fragile]
\frametitle{NAMESPACE}
\label{sec-2-6}

\begin{itemize}
\item Para registar el método \texttt{S3} \texttt{print} para la clase \texttt{myClass}:
\end{itemize}

\lstset{language=R}
\begin{lstlisting}
S3method(print, foo)
\end{lstlisting}
\begin{itemize}
\item Para registrar las clases \texttt{S4} \texttt{class1} y \texttt{class2}:
\end{itemize}

\lstset{language=R}
\begin{lstlisting}
exportClasses(class1, class2)
\end{lstlisting}
\begin{itemize}
\item Para registrar los métodos \texttt{S4} \texttt{method1} y \texttt{method2}:
\end{itemize}

\lstset{language=R}
\begin{lstlisting}
exportMethods(method1, method2)
\end{lstlisting}
\begin{itemize}
\item Para importar métodos y clases \texttt{S4} de otro paquete:
\end{itemize}

\lstset{language=R}
\begin{lstlisting}
importClassesFrom(package, ...)
importMethodsFrom(package, ...)
\end{lstlisting}
\end{frame}
\begin{frame}
\frametitle{Documentación}
\label{sec-2-7}

\begin{itemize}
\item Las páginas de ayuda de los objetos \texttt{R} se escriben usando el
  formato “R documentation” (Rd), un lenguaje similar a \LaTeX{}.
\item Es aconsejable seguir estas orientaciones: \href{http://developer.r-project.org/Rds.html}{Guidelines for Rd files}
\item Para generar el esqueleto de un fichero Rd es aconsejable usar:
\begin{itemize}
\item \texttt{prompt}: \href{http://cran.r-project.org/doc/manuals/r-release/R-exts.html\#Documenting-functions}{genérica}
\item \texttt{promptClass} y \texttt{promptMethods}: \href{http://cran.r-project.org/doc/manuals/r-release/R-exts.html\#Documenting-S4-classes-and-methods}{clases y métodos}.
\item \texttt{promptPackage}: \href{http://cran.r-project.org/doc/manuals/r-release/R-exts.html\#Documenting-packages}{paquete}
\item \texttt{promptData}: \href{http://cran.r-project.org/doc/manuals/r-release/R-exts.html\#Documenting-data-sets}{datos}
\end{itemize}
\item Todos los comandos disponibles están en el documento \href{http://developer.r-project.org/parseRd.pdf}{Parsing Rd files}.
\end{itemize}
\end{frame}
\begin{frame}[fragile]


\begin{verbatim}
\name{load}
\alias{load}
\title{Reload Saved Datasets}
\description{
  Reload the datasets written to a file with the function
  \code{save}.
}
\usage{
  load(file, envir = parent.frame())
}
\arguments{
\item{file}{a connection or a character string giving the
    name of the file to load.}
\item{envir}{the environment where the data should be
    loaded.}
}
\seealso{
  \code{\link{save}}.
}
\examples{
  ## save all data
  save(list = ls(), file= "all.RData")
  
  ## restore the saved values to the current environment
  load("all.RData")
  
  ## restore the saved values to the workspace
  load("all.RData", .GlobalEnv)
}
\keyword{file}
\end{verbatim}
\end{frame}
\section{Publicar un paquete}
\label{sec-3}
\begin{frame}[fragile]
\frametitle{Itinerario}
\label{sec-3-1}

\begin{itemize}
\item Comprobar
\end{itemize}

\begin{verbatim}
R CMD check myPackage/
\end{verbatim}
\begin{itemize}
\item Construir
\end{itemize}

\begin{verbatim}
R CMD build myPackage/
\end{verbatim}
\begin{itemize}
\item Publicar (o actualizar) en un repositorio
\end{itemize}
\end{frame}
\begin{frame}[fragile]
\frametitle{Comprobar}
\label{sec-3-2}

\begin{itemize}
\item Comprobar un directorio (desde línea de comandos):
\end{itemize}

\begin{verbatim}
R CMD check myPackage/
\end{verbatim}
\begin{itemize}
\item Comprobar un paquete ya construido (desde línea de comandos):
\end{itemize}

\begin{verbatim}
R CMD check myPackage.tar.gz
\end{verbatim}
\begin{itemize}
\item Esta comprobación incluye más de 20 puntos de prueba detallados
  en el manual \href{http://cran.r-project.org/doc/manuals/R-exts.html#Checking-packages}{Writing R extensions}.
\end{itemize}
\end{frame}
\begin{frame}[fragile]
\frametitle{Construir}
\label{sec-3-3}
\begin{block}{Fuente o binario}
\label{sec-3-3-1}

    Se puede construir un fichero fuente en formato \emph{tarball}
    (independiente de la plataforma, habitual en sistemas Unix) o en
    forma binaria (dependiente de la plataforma, habitual para Windows y Mac).
\end{block}
%% Cómo hacerlo
\label{sec-3-3-2}

\begin{itemize}
\item Fuente en formato \emph{tarball}
\begin{itemize}
\item El resultado es un fichero \emph{tarball} \texttt{myPackage.tar.gz} que
     se puede distribuir a cualquier sistema.
\end{itemize}
\end{itemize}

\begin{verbatim}
R CMD build myPackage/
\end{verbatim}
\begin{itemize}
\item Comprimido binario
\begin{itemize}
\item El resultado es una copia comprimida de la versión
     \textbf{instalada} del paquete: depende del sistema operativo.
\end{itemize}
\end{itemize}

\begin{verbatim}
R CMD INSTALL --build myPackage/
\end{verbatim}
\end{frame}
\begin{frame}[fragile]
\frametitle{Comprobar y construir en sistemas Windows}
\label{sec-3-4}


\begin{itemize}
\item Para paquetes sin código compilado (C, Fortran), también se puede usar
  \texttt{R CMD check} y \texttt{R CMD build} en un sistema Windows.
\item Para generar un binario hay que usar \texttt{R CMD INSTALL -{}-build}.
\begin{itemize}
\item Es posible que haya que modificar la variables de entorno
    \texttt{TEMP} y \texttt{TMP} de forma que \textbf{sólo} contengan caracteres ASCII.
\end{itemize}
\item Para paquetes con código compilado, o en caso de problemas con
  los comandos anteriores, hay que usar \href{http://cran.r-project.org/bin/windows/Rtools/}{Rtools}.
\item Se pueden instalar fuentes \emph{tarball} con (ver \href{http://cran.r-project.org/doc/manuals/R-admin.html#Windows-packages}{R installation and administration}):
\end{itemize}

\begin{verbatim}
install.packages(myPackage.tar.gz, type='source')
\end{verbatim}
\end{frame}
\begin{frame}
\frametitle{Repositorios}
\label{sec-3-5}

\begin{itemize}
\item El principal repositorio de paquetes estables es \href{http://cran.r-project.org/}{CRAN}.
\begin{itemize}
\item Publicar en este repositorio conlleva la aceptación de unas \href{http://cran.r-project.org/web/packages/policies.html}{condiciones}.
\item Para publicar en CRAN hay que subir el fichero fuente
    \emph{tarball} resultado de \texttt{R CMD build} via FTP anónimo a la
    dirección \href{ftp://CRAN.R-project.org/incoming/}{ftp://CRAN.R-project.org/incoming/} y enviar un
    correo en texto plano a CRAN@R-project.org.
\item Es imprescindible haber comprobado el fichero con \texttt{R CMD check     -{}-as-cran} antes de subirlo al servidor FTP. El resultado de
    esta comprobación \textbf{no} debe contener errores ni advertencias
    (\texttt{warnings}).
\item Más detalle en el apartado Submission de las \href{http://cran.r-project.org/web/packages/policies.html}{condiciones de CRAN} y en el manual \href{http://cran.r-project.org/doc/manuals/R-exts.html\#Submitting-a-package-to-CRAN}{R Extensions}.
\end{itemize}
\end{itemize}
\end{frame}
\begin{frame}
\frametitle{Repositorios}
\label{sec-3-6}


\begin{itemize}
\item Otros repositorios destacables son:
\begin{itemize}
\item \href{http://r-forge.r-project.org/}{R-Forge} (versiones de desarrollo)
\item \href{http://www.bioconductor.org/}{Bioconductor} (paquetes de bioinformática)
\item \href{https://github.com/languages/R}{Proyectos R en Github} (versiones de desarrollo)
\item \href{http://rforge.net/}{RForge} (versiones de desarrollo)
\end{itemize}
\end{itemize}
\end{frame}

\end{document}