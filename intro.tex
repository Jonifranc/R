% Created 2013-02-21 jue 12:21
\documentclass[xcolor={usenames,svgnames,dvipsnames}]{beamer}
\usepackage[utf8]{inputenc}
\usepackage[T1]{fontenc}
\usepackage{fixltx2e}
\usepackage{graphicx}
\usepackage{longtable}
\usepackage{float}
\usepackage{wrapfig}
\usepackage{soul}
\usepackage{textcomp}
\usepackage{marvosym}
\usepackage{wasysym}
\usepackage{latexsym}
\usepackage{amssymb}
\usepackage{hyperref}
\tolerance=1000
\usepackage{color}
\usepackage{listings}
\AtBeginSubsection[]{\begin{frame}[plain]\tableofcontents[currentsubsection]\end{frame}}
\lstset{commentstyle=\color{gray!90}, basicstyle=\ttfamily\small, columns=fullflexible, breaklines=true,linewidth=\textwidth, backgroundcolor=\color{gray!23}, basewidth={0.5em,0.4em}, literate={á}{{\'a}}1 {ñ}{{\~n}}1 {é}{{\'e}}1 {ó}{{\'o}}1 {º}{{\textordmasculine}}1}
\usepackage{mathpazo}
\setbeamercovered{transparent}
\usefonttheme{serif} 
\usetheme{Goettingen}
\hypersetup{colorlinks=true, linkcolor=Blue, urlcolor=Blue}
\providecommand{\alert}[1]{\textbf{#1}}

\title{Introducción a R}
\author{Oscar Perpiñán Lamigueiro}
\date{30 de Enero de 2013}
\hypersetup{
  pdfkeywords={},
  pdfsubject={},
  pdfcreator={Emacs Org-mode version 7.8.11}}

\begin{document}

\maketitle


\section{R es software libre}
\label{sec-1}
\subsection{¿Qué es \texttt{R}?}
\label{sec-1-1}
\begin{frame}
\frametitle{¿Qué es \texttt{R}?}
\label{sec-1-1-1}

¿Qué es \href{http://procomun.wordpress.com/2011/02/23/que-es-r/}{R}?: es un lenguaje de programación principalmente
orientado al análisis estadístico y visualización de información
cuantitativa y cualitativa y publicado como software libre con
licencia GNU-GPL.
\end{frame}
\subsection{R es un proyecto colaborativo}
\label{sec-1-2}
\begin{frame}
\frametitle{R es un proyecto colaborativo}
\label{sec-1-2-1}

\begin{itemize}
\item Una de las grandes riquezas de R es la cantidad de paquetes (más
  de 4000 actualmente) que amplían sus funcionalidades.
\item La lista completa está en \href{http://cran.es.r-project.org/web/packages/}{http://cran.es.r-project.org/web/packages/}.
\item Las CRAN Task Views agrupan por temáticas:
  \href{http://cran.r-project.org/web/views/}{http://cran.r-project.org/web/views/}
\end{itemize}
\end{frame}
\begin{frame}[fragile]
\frametitle{Más de 4000 paquetes disponibles}
\label{sec-1-2-2}

\begin{itemize}
\item Algunos vienen instalados y se cargan al empezar:
\end{itemize}

\lstset{language=R}
\begin{lstlisting}
sessionInfo()
\end{lstlisting}
\begin{itemize}
\item Otros vienen instalados pero hay que cargarlos:
\end{itemize}

\lstset{language=R}
\begin{lstlisting}
library(lattice)
packageVersion('lattice')
packageDescription('lattice')
\end{lstlisting}
\begin{itemize}
\item Otros hay que instalarlos y después cargarlos:
\end{itemize}

\lstset{language=R}
\begin{lstlisting}
install.packages('zoo')
library('zoo')
packageDescription('zoo')
\end{lstlisting}
\end{frame}
\subsection{R está muy bien documentado}
\label{sec-1-3}
\begin{frame}
\frametitle{Fuentes de información}
\label{sec-1-3-1}

\begin{itemize}
\item \href{http://cran.r-project.org/doc/manuals/R-intro.html}{Manual introductorio}
\item \href{http://cran.r-project.org/manuals.html}{Manuales oficiales}
\item \href{http://cran.r-project.org/other-docs.html}{Manuales externos}
\item \href{http://www.r-project.org/mail.html}{Listas de correo} (sin olvidar respetar \href{http://www.r-project.org/posting-guide.html}{estos consejos}).
\item \href{http://www.r-bloggers.com}{R-bloggers}
\end{itemize}
\end{frame}
\section{Vectores y Matrices}
\label{sec-2}
\subsection{Vectores}
\label{sec-2-1}
\begin{frame}[fragile]
\frametitle{Primeros pasos}
\label{sec-2-1-1}


\lstset{language=R}
\begin{lstlisting}
x <- 1
x
length(x)
class(x)

x <- c(1, 2, 3)
x
length(x)
class(x)
\end{lstlisting}
\end{frame}
\begin{frame}[fragile]
\frametitle{Primeras funciones}
\label{sec-2-1-2}


\lstset{language=R}
\begin{lstlisting}
class(c)
class(length)
length
\end{lstlisting}
\end{frame}
\begin{frame}[fragile]
\frametitle{Operaciones sencillas con vectores}
\label{sec-2-1-3}


\lstset{language=R}
\begin{lstlisting}
x + 1
y <- 1:10
x + y
x * y
x^2
x^2 + y^3
exp(x)
log(x)
\end{lstlisting}
\end{frame}
\begin{frame}[fragile]
\frametitle{Generar vectores con \texttt{seq}}
\label{sec-2-1-4}


\lstset{language=R}
\begin{lstlisting}
x1 <- seq(1, 100, by=2)
x1
help(seq)

seq(1, 100, 10)
seq(1, 100, length=10)
seq(1, 1, 10)

x <- seq(1, 100, length=10)
x
length(x)

x <- seq(1, 100, length=10)
y <- seq(2, 100, length=50)
\end{lstlisting}
\end{frame}
\begin{frame}[fragile]
\frametitle{Unir vectores con \texttt{c}}
\label{sec-2-1-5}


\lstset{language=R}
\begin{lstlisting}
z <- c(x, y)
z
z + c(1, 2)
z + c(1, 2, 3, 4, 5, 6, 7)
z <- c(z, z, z, z)
z
\end{lstlisting}
\end{frame}
\begin{frame}[fragile]
\frametitle{Generar vectores con \texttt{rep}}
\label{sec-2-1-6}


\lstset{language=R}
\begin{lstlisting}
rep(1:10, 4)

length(z)

rep(c(1, 2, 3), 10)
rep(c(1, 2, 3), each=10)
help(rep)
\end{lstlisting}
\end{frame}
\begin{frame}[fragile]
\frametitle{Indexado numérico de vectores}
\label{sec-2-1-7}



\lstset{language=R}
\begin{lstlisting}
x <- seq(1, 100, 2)
1:5
x[c(1, 2, 3, 4, 5)]
x[1:5]
x[10:5]
\end{lstlisting}
\end{frame}
\begin{frame}[fragile]
\frametitle{Indexado de vectores con condiciones lógicas}
\label{sec-2-1-8}


\lstset{language=R}
\begin{lstlisting}
condicion <- (x>30)
condicion
class(condicion)
\end{lstlisting}
\end{frame}
\begin{frame}[fragile]
\frametitle{Indexado de vectores con condiciones lógicas}
\label{sec-2-1-9}



\lstset{language=R}
\begin{lstlisting}
x==37
x[x==37]
x[x!=9]
x[x>20]
\end{lstlisting}
\begin{block}{Y aquí, ¿qué ocurre?}
\label{sec-2-1-9-1}



\lstset{language=R}
\begin{lstlisting}
x[x=10]
\end{lstlisting}
     
\end{block}
\end{frame}
\begin{frame}[fragile]
\frametitle{Indexado de vectores con \texttt{\%in\%}}
\label{sec-2-1-10}


\lstset{language=R}
\begin{lstlisting}
y <- seq(101, 200, 2)
y %in% c(101, 127, 141)
y
y[y %in% c(101, 127, 141)]
\end{lstlisting}
\end{frame}
\begin{frame}[fragile]
\frametitle{Indexado de vectores con condiciones múltiples}
\label{sec-2-1-11}



\lstset{language=R}
\begin{lstlisting}
z <- c(x, y)
z
z>150
z[z>150]
z[z<30 | z>150]
z[z>=30 & z<=150]
z[c(1, 10, 40, 80)]
\end{lstlisting}
\end{frame}
\begin{frame}[fragile]
\frametitle{Indexado de vectores con condiciones múltiples}
\label{sec-2-1-12}


\lstset{language=R}
\begin{lstlisting}
cond  <-  (x>10) & (x<50)
cond
cond  <-  (x>=10) & (x<=50)
cond
x[cond]
\end{lstlisting}
\end{frame}
\begin{frame}[fragile]
\frametitle{Con las condiciones se pueden hacer operaciones}
\label{sec-2-1-13}


\lstset{language=R}
\begin{lstlisting}
sum(cond)
!cond
sum(!cond)
length(x[cond])
length(x[!cond])
as.numeric(cond)
\end{lstlisting}
    
\end{frame}
\begin{frame}[fragile]
\frametitle{Aritmética sencilla}
\label{sec-2-1-14}


\lstset{language=R}
\begin{lstlisting}
x + y
x - y
x * y
x^2 + y^3
exp(x)
log(x)
\end{lstlisting}
\end{frame}
\begin{frame}[fragile]
\frametitle{Funciones predefinidas}
\label{sec-2-1-15}


\lstset{language=R}
\begin{lstlisting}
summary(x)
mean(x)
sd(x)
median(x)
max(x)
min(x)
range(x)
quantile(x)
\end{lstlisting}
\end{frame}
\begin{frame}[fragile]
\frametitle{Funciones y condiciones}
\label{sec-2-1-16}


\lstset{language=R}
\begin{lstlisting}
sum(x)
sum(x[cond])
sum(x[(x>=10) & (x<=50)])
x[1] + x[2] + x[3] + x[4] + x[5]
sum(x[1:5])
\end{lstlisting}
\end{frame}
\begin{frame}[fragile]
\frametitle{¿Y qué hago cuando necesito ayuda?}
\label{sec-2-1-17}



\lstset{language=R}
\begin{lstlisting}
help(exp)
help(sum)
help(quantile)
\end{lstlisting}
\end{frame}
\subsection{Matrices}
\label{sec-2-2}
\begin{frame}[fragile]
\frametitle{Construir una matriz}
\label{sec-2-2-1}


\lstset{language=R}
\begin{lstlisting}
z <- 1:12
M  <-  matrix(z, nrow=3)
M
z
help(matrix)
class(M)
dim(M)
summary(M)
\end{lstlisting}
\end{frame}
\begin{frame}[fragile]
\frametitle{Matrices a partir de vectores: \texttt{rbind} y \texttt{cbind}}
\label{sec-2-2-2}


\lstset{language=R}
\begin{lstlisting}
x <- 1:10
y <- 1:10
z <- 1:10
z <- y <- x <- 1:10

M <- cbind(x, y, z)
M
M <- rbind(x, y, z)
M

rbind(M, M)
cbind(M, M)
\end{lstlisting}
\end{frame}
\begin{frame}[fragile]
\frametitle{Transponer una matriz}
\label{sec-2-2-3}



\lstset{language=R}
\begin{lstlisting}
t(M)
class(t)
dim(t(M))
\end{lstlisting}
\end{frame}
\begin{frame}[fragile]
\frametitle{Operaciones con matrices}
\label{sec-2-2-4}



\lstset{language=R}
\begin{lstlisting}
M * M
M ^ 2
M %*% M
M %*% t(M)
help('%*%')
\end{lstlisting}
\end{frame}
\begin{frame}[fragile]
\frametitle{Operaciones con matrices: funciones predefinidas}
\label{sec-2-2-5}



\lstset{language=R}
\begin{lstlisting}
sum(M)
rowSums(M)
colSums(M)
rowMeans(M)
colMeans(M)
\end{lstlisting}
\end{frame}
\begin{frame}[fragile]
\frametitle{La función \texttt{apply}}
\label{sec-2-2-6}


\lstset{language=R}
\begin{lstlisting}
help(apply)
apply(M, 1, sum)
apply(M, 2, sum)
apply(M, 1, mean)
apply(M, 2, mean)
apply(M, 1, sd, na.rm=TRUE)
apply(M, 2, sd)
\end{lstlisting}
\end{frame}
\begin{frame}[fragile]
\frametitle{\texttt{sweep}}
\label{sec-2-2-7}

\begin{itemize}
\item Usamos el conjunto de datos \texttt{state.x77}
\end{itemize}

\lstset{language=R}
\begin{lstlisting}
head(state.x77)
\end{lstlisting}
\begin{itemize}
\item Calculamos el máximo por columna
\end{itemize}

\lstset{language=R}
\begin{lstlisting}
maxes <- apply(state.x77, 2, max)
\end{lstlisting}
\begin{itemize}
\item Dividimos cada columna por su máximo
\end{itemize}

\lstset{language=R}
\begin{lstlisting}
stateNorm <- sweep(state.x77, 2, maxes, FUN="/")
head(stateNorm)
\end{lstlisting}
\end{frame}
\begin{frame}[fragile]
\frametitle{Indexado de matrices}
\label{sec-2-2-8}


\lstset{language=R}
\begin{lstlisting}
M
M[]
M[1, ]
M[, 1]
sum(M[, 1])
M[1:2, ]
M[1:2, 2:3]
M[1, c(1, 4)]
M[-1,]
M[-c(1, 2),]
\end{lstlisting}
    
\end{frame}
\subsection{Valores ausentes}
\label{sec-2-3}

   
\begin{frame}[fragile]
\frametitle{¿Qué es \texttt{NA}?}
\label{sec-2-3-1}


\lstset{language=R}
\begin{lstlisting}
class(NA)
x <- rnorm(100)
idx <- sample(length(x), 10)
idx
x[idx]
x2 <- x
x2[idx] <- NA
x2
\end{lstlisting}
\end{frame}
\begin{frame}[fragile]
\frametitle{\texttt{NA} en las funciones}
\label{sec-2-3-2}



\lstset{language=R}
\begin{lstlisting}
summary(x)
mean(x)
sum(x)

summary(x2)
mean(x2)
sum(x2)
\end{lstlisting}
\end{frame}
\begin{frame}[fragile]
\frametitle{\texttt{NA} en las funciones}
\label{sec-2-3-3}



\lstset{language=R}
\begin{lstlisting}
mean(x2, na.rm=TRUE)
sum(x2, na.rm=TRUE)
sd(x2, na.rm=TRUE)
class(TRUE)
\end{lstlisting}
\end{frame}
\section{Funciones}
\label{sec-3}
\subsection{Definición de funciones}
\label{sec-3-1}
\begin{frame}[fragile]
\frametitle{Para definir una función usamos la función \texttt{function}}
\label{sec-3-1-1}


\lstset{language=R}
\begin{lstlisting}
myFun <- function(x, y) x + y
myFun(3, 4)
class(myFun)
\end{lstlisting}
\end{frame}
\begin{frame}[fragile]
\frametitle{Definir una función a partir de funciones}
\label{sec-3-1-2}


\lstset{language=R}
\begin{lstlisting}
foo  <-  function(x, ...){
  mx <- mean(x, ...)
  medx <- median(x, ...)
  sdx <- sd(x, ...)
  c(mx, medx, sdx)
  }
\end{lstlisting}
O en forma resumida:

\lstset{language=R}
\begin{lstlisting}
foo <- function(x, ...){c(mean(x, ...), median(x, ...), sd(x, ...))}
\end{lstlisting}
\end{frame}
\subsection{Uso de funciones}
\label{sec-3-2}
\begin{frame}[fragile]
\frametitle{Y ahora usamos la función con vectores}
\label{sec-3-2-1}


\lstset{language=R}
\begin{lstlisting}
foo(1:10)

rnorm(100)
help(rnorm)
foo(rnorm(1e5))
\end{lstlisting}
\end{frame}
\begin{frame}[fragile]
\frametitle{Y también funciona con matrices}
\label{sec-3-2-2}


\lstset{language=R}
\begin{lstlisting}
rowMeans(M)
apply(M, 1, foo)
colMeans(M)
apply(M, 2, foo)
\end{lstlisting}
\end{frame}
\begin{frame}[fragile]
\frametitle{La función \texttt{outer}}
\label{sec-3-2-3}


\lstset{language=R}
\begin{lstlisting}
f <- function(x, y)x^2+y^2
f
f(1, 2)
x
y

z <- outer(x, y, f)
z
image(x, y, z)
\end{lstlisting}
\end{frame}
\section{Listas y data.frame}
\label{sec-4}
\subsection{Listas}
\label{sec-4-1}
\begin{frame}[fragile]
\frametitle{Para crear una lista usamos la función \texttt{list}}
\label{sec-4-1-1}


\lstset{language=R}
\begin{lstlisting}
lista <- list(a=c(1,3,5),
              b=c('l', 'p', 'r', 's'),
              c=3)
class(list)
class(lista)
\end{lstlisting}
\end{frame}
\begin{frame}[fragile]
\frametitle{Podemos acceder a los elementos\ldots{}}
\label{sec-4-1-2}

\begin{itemize}
\item Por su nombre
\end{itemize}

\lstset{language=R}
\begin{lstlisting}
lista
lista$a
lista$b
lista$c
\end{lstlisting}

\begin{itemize}
\item o por su índice
\end{itemize}

\lstset{language=R}
\begin{lstlisting}
lista[1]
lista[[1]]

class(lista[1])
class(lista[[1]])

lista[2]
lista[[2]]

class(lista[2])
class(lista[[2]])
\end{lstlisting}
\end{frame}
\begin{frame}[fragile]
\frametitle{Cada elemento es diferente}
\label{sec-4-1-3}

\begin{itemize}
\item Clase
\end{itemize}

\lstset{language=R}
\begin{lstlisting}
class(lista)
class(lista$a)
class(lista$b)
class(lista$c)
\end{lstlisting}
\begin{itemize}
\item Longitud
\end{itemize}

\lstset{language=R}
\begin{lstlisting}
length(lista)
length(lista$a)
length(lista$b)
length(lista$c)
\end{lstlisting}
\end{frame}
\begin{frame}[fragile]
\frametitle{Para matrices \texttt{apply}, para listas \texttt{lapply} y \texttt{sapply}}
\label{sec-4-1-4}


\lstset{language=R}
\begin{lstlisting}
lapply(lista, length)
sapply(lista, length)

lista <- list(x = 1:10,
              y = seq(0, 10, 2),
              z = rnorm(30))
lista

lapply(lista, sum)
lapply(lista, median)
lapply(lista, foo)
\end{lstlisting}
\end{frame}
\subsection{Data.frame}
\label{sec-4-2}
\begin{frame}[fragile]
\frametitle{Para crear un \texttt{data.frame}...}
\label{sec-4-2-1}


\lstset{language=R}
\begin{lstlisting}
df <- data.frame(x = 1:10,
                 y = rnorm(10),
                 z = 0)

length(df)
dim(df)
\end{lstlisting}
\end{frame}
\begin{frame}[fragile]
\frametitle{Podemos acceder a los elementos}
\label{sec-4-2-2}

\begin{itemize}
\item Por su nombre
\end{itemize}

\lstset{language=R}
\begin{lstlisting}
df$x
df$y
df$z
\end{lstlisting}

\begin{itemize}
\item Por su índice
\end{itemize}

\lstset{language=R}
\begin{lstlisting}
df
df[1,]
df[,1]
df[,2]
\end{lstlisting}
\end{frame}
\begin{frame}[fragile]
\frametitle{La regla del reciclaje}
\label{sec-4-2-3}


\lstset{language=R}
\begin{lstlisting}
year <- 2011
month <- 1:12
class <- c('A', 'B', 'C')
vals <- rnorm(12)

dats <- data.frame(year, month, class, vals)
dats
\end{lstlisting}
\end{frame}
\begin{frame}[fragile]
\frametitle{La función \texttt{expand.grid}}
\label{sec-4-2-4}



\lstset{language=R}
\begin{lstlisting}
x <- y <- seq(-4*pi, 4*pi, len=200)
df <- expand.grid(x = x, y = y)
head(df)
tail(df)
summary(df)
dim(df)
names(df)
\end{lstlisting}
\end{frame}
\begin{frame}[fragile]
\frametitle{Funciones sobre \texttt{data.frame}}
\label{sec-4-2-5}



\lstset{language=R}
\begin{lstlisting}
circles <- function(object){
  r <- with(object, sqrt(x^2 + y^2))
  res <- cos(r^2)*exp(-r/6)
  res}

df$result <- circles(df)
head(df)
\end{lstlisting}
\end{frame}
\begin{frame}[fragile]
\frametitle{Una imagen vale más que mil palabras}
\label{sec-4-2-6}


\lstset{language=R}
\begin{lstlisting}
library(lattice)
levelplot(result ~ x + y, data=df)
\end{lstlisting}
\end{frame}
\begin{frame}[fragile]
\frametitle{Unir dos \texttt{data.frame}}
\label{sec-4-2-7}

\begin{itemize}
\item Primero construimos un \texttt{data.frame} de ejemplo
\end{itemize}

\lstset{language=R}
\begin{lstlisting}
USStates <- as.data.frame(state.x77)
USStates$Name <- rownames(USStates)
rownames(USStates) <- NULL
\end{lstlisting}
\begin{itemize}
\item Lo partimos en estados ``fríos'' y estados ``grandes''
\end{itemize}

\lstset{language=R}
\begin{lstlisting}
coldStates <- USStates[USStates$Frost>150, c('Name', 'Frost')]
largeStates <- USStates[USStates$Area>1e5, c('Name', 'Area')]
\end{lstlisting}
\begin{itemize}
\item Unimos los dos conjuntos (estados ``fríos'' y ``grandes'')
\end{itemize}

\lstset{language=R}
\begin{lstlisting}
merge(coldStates, largeStates)
\end{lstlisting}
\end{frame}
\begin{frame}[fragile]
\frametitle{\texttt{merge} usa \texttt{match}}
\label{sec-4-2-8}

\begin{itemize}
\item Estados grandes que también son fríos
\end{itemize}

\lstset{language=R}
\begin{lstlisting}
idxLarge <- match(largeStates$Name,
                  coldStates$Name,
                  nomatch=0)
idxLarge

largeStates[idxLarge,]
\end{lstlisting}

\begin{itemize}
\item Estados frios que también son grandes
\end{itemize}

\lstset{language=R}
\begin{lstlisting}
idxCold <- match(coldStates$Name,
                 largeStates$Name,
                 nomatch=0)
idxCold

coldStates[idxCold,]
\end{lstlisting}
\end{frame}
\section{Factores, fechas y caracteres}
\label{sec-5}
\subsection{\texttt{factor}}
\label{sec-5-1}
\begin{frame}[fragile]
\frametitle{Una variable numérica que nos servirá para el ejemplo}
\label{sec-5-1-1}


\lstset{language=R}
\begin{lstlisting}
N <- 100
edad <- sample(seq(18, 40, 1), N, replace=TRUE)
summary(edad)
\end{lstlisting}
\end{frame}
\begin{frame}[fragile]
\frametitle{Una variable cualitativa se define con \texttt{factor}}
\label{sec-5-1-2}

\begin{itemize}
\item Ahora es un \texttt{character}
\end{itemize}

\lstset{language=R}
\begin{lstlisting}
sexo <- sample(c('H', 'M'), N, replace=TRUE)
class(sexo)
summary(sexo)
\end{lstlisting}
\begin{itemize}
\item Ahora es un \texttt{factor}
\end{itemize}

\lstset{language=R}
\begin{lstlisting}
sexo <- factor(sexo)
class(sexo)
summary(sexo)
levels(sexo)
nlevels(sexo)
\end{lstlisting}
\end{frame}
\begin{frame}[fragile]
\frametitle{Los \texttt{factor} sirven para agrupar}
\label{sec-5-1-3}


\begin{itemize}
\item Con la función \texttt{table}
\end{itemize}

\lstset{language=R}
\begin{lstlisting}
table(edad, sexo)
table(edad > 30, sexo)
table(edad %in% 20:30, sexo)
\end{lstlisting}

\begin{itemize}
\item Con \texttt{tapply} o \texttt{aggregate}
\end{itemize}

\lstset{language=R}
\begin{lstlisting}
tapply(edad,sexo, mean)
aggregate(edad ~ sexo, FUN=median)
\end{lstlisting}
\end{frame}
\begin{frame}[fragile]
\frametitle{Los factores sirven para separar}
\label{sec-5-1-4}


\lstset{language=R}
\begin{lstlisting}
edadSexo <- split(edad, sexo)
class(edadSexo)

sapply(edadSexo, mean)
\end{lstlisting}
\end{frame}
\begin{frame}[fragile]
\frametitle{Los \texttt{factor} se pueden generar a partir de variables numéricas}
\label{sec-5-1-5}

\begin{itemize}
\item Por ejemplo, con \texttt{cut}
\end{itemize}

\lstset{language=R}
\begin{lstlisting}
gEdad <- cut(edad, breaks=4)
class(gEdad)
levels(gEdad)
nlevels(gEdad)
\end{lstlisting}

\begin{itemize}
\item Nuevamente \texttt{table}
\end{itemize}

\lstset{language=R}
\begin{lstlisting}
table(gEdad)
table(gEdad, sexo)
\end{lstlisting}
\end{frame}
\subsection{Fechas}
\label{sec-5-2}
\begin{frame}[fragile]
\frametitle{\texttt{Date}}
\label{sec-5-2-1}


\lstset{language=R}
\begin{lstlisting}
as.Date('2013-02-06')
as.Date('2013/02/06')

as.Date('06.02.2013')
as.Date('06.02.2013', format='%d.%m.%Y')

as.Date(37, origin='2013-01-01')
\end{lstlisting}
\end{frame}
\begin{frame}[fragile]
\frametitle{Secuencias temporales con \texttt{Date}}
\label{sec-5-2-2}


\lstset{language=R}
\begin{lstlisting}
seq(as.Date('2004-01-01'), by='day', length=10)
seq(as.Date('2004-01-01'), by='month', length=10)
seq(as.Date('2004-01-01'), by='10 day', length=10)
\end{lstlisting}
\end{frame}
\begin{frame}[fragile]
\frametitle{POSIXct}
\label{sec-5-2-3}


\lstset{language=R}
\begin{lstlisting}
as.POSIXct('2013-02-06')
ISOdate(2013, 2, 7)
\end{lstlisting}


\lstset{language=R}
\begin{lstlisting}
hoy <- as.POSIXct('2013-02-06')

help(format.POSIXct)
format(hoy, '%Y')
format(hoy, '%d')
format(hoy, '%m')
format(hoy, '%b')
format(hoy, '%d de %B de %Y')
\end{lstlisting}


\lstset{language=R}
\begin{lstlisting}
hora <- Sys.time()
hora

format(hora, '%H:%M:%S')
format(hora, '%H horas, %M minutos y %S segundos')
\end{lstlisting}
\end{frame}
\begin{frame}[fragile]
\frametitle{Secuencias temporales con \texttt{POSIXct}}
\label{sec-5-2-4}


\lstset{language=R}
\begin{lstlisting}
seq(as.POSIXct('2004-01-01'), by='month', length=10)
seq(as.POSIXct('2004-01-01 10:00:00'), by='15 min', length=10)
\end{lstlisting}
\end{frame}
\begin{frame}[fragile]
\frametitle{Zonas horarias}
\label{sec-5-2-5}



\lstset{language=R}
\begin{lstlisting}
as.POSIXct('2013-02-06 15:30:00', tz='GMT')
as.POSIXct('2013-02-06 15:30:00', tz='Europe/Madrid')
\end{lstlisting}


\lstset{language=R}
\begin{lstlisting}
hawaii <- as.POSIXct('2013-02-06 15:30:00', tz='HST')
## Character
format(hawaii, tz='GMT')
## POSIXct
as.POSIXct(format(hawaii, tz='GMT'), tz='GMT')
\end{lstlisting}
\end{frame}
\subsection{Caracteres}
\label{sec-5-3}
\begin{frame}[fragile]
\frametitle{Bastan unas simples comillas}
\label{sec-5-3-1}



\lstset{language=R}
\begin{lstlisting}
cadena <- "Hola mundo"
class(cadena)
nchar(cadena)
\end{lstlisting}

\begin{itemize}
\item Y aquí, ¿qué pasa?
\end{itemize}

\lstset{language=R}
\begin{lstlisting}
length(cadena)
cadena[1]
cadena[2]
\end{lstlisting}
\end{frame}
\begin{frame}[fragile]
\frametitle{Un vector de \texttt{character}}
\label{sec-5-3-2}


\lstset{language=R}
\begin{lstlisting}
cadenaVec <- c("Hola mundo", "Hello world")
nchar(cadenaVec)
length(cadenaVec)
\end{lstlisting}
\end{frame}
\begin{frame}[fragile]
\frametitle{Para mostrarlos usamos \texttt{cat} o \texttt{print}}
\label{sec-5-3-3}



\lstset{language=R}
\begin{lstlisting}
a = 2
b = 3

cat('La suma de', a, 'y', b, 'es', a + b)

cat('La suma de', a, 'y', b, 'es', a + b, fill=TRUE)

cat('La suma de', a, 'y', b, 'es', a + b, '\n',
    'La multiplicación de', a, 'por', b, 'es', a*b, '\n')

cat('La suma de', a, 'y', b, 'es', a + b, '\n',
    'La multiplicación de', a, 'por', b, 'es', a*b, fill=15)
\end{lstlisting}
\end{frame}
\begin{frame}[fragile]
\frametitle{Los \texttt{character} se pueden unir\ldots{}}
\label{sec-5-3-4}

\begin{itemize}
\item Primero sencillo
\end{itemize}

\lstset{language=R}
\begin{lstlisting}
paste('Hello', 'World', sep='_')

paste(cadenaVec)
paste(cadenaVec, collapse='=')
\end{lstlisting}
\begin{itemize}
\item Y algo más complicado
\end{itemize}

\lstset{language=R}
\begin{lstlisting}
paste('X', 1:5, sep='.')
paste(c('A', 'B'), 1:5, sep='.')

paste(c('A', 'B'), 1:5, sep='.', collapse='|')
\end{lstlisting}
\end{frame}
\begin{frame}[fragile]
\frametitle{\ldots{} y también se pueden separar\ldots{}}
\label{sec-5-3-5}


\lstset{language=R}
\begin{lstlisting}
strsplit(cadenaVec, split=' ')
strsplit(cadenaVec, split='')
\end{lstlisting}


\lstset{language=R}
\begin{lstlisting}
chSep <- strsplit(cadenaVec, split=' ')
class(chSep)
length(chSep)
sapply(chSep, length)
sapply(chSep, nchar)
\end{lstlisting}
\end{frame}
\begin{frame}[fragile]
\frametitle{\ldots{} y, por supuesto, manipular}
\label{sec-5-3-6}


\lstset{language=R}
\begin{lstlisting}
sub('o', '0', 'Hola Mundo')
gsub('o', '0', 'Hola Mundo')

substring(cadena, 1) <- 'HOLA'
cadena

tolower(cadena)
toupper(cadena)
\end{lstlisting}
\end{frame}

\end{document}