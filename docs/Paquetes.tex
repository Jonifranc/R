% Created 2017-06-29 jue 10:48
% Intended LaTeX compiler: pdflatex
\documentclass[xcolor={usenames,svgnames,dvipsnames}]{beamer}
\usepackage[utf8]{inputenc}
\usepackage[T1]{fontenc}
\usepackage{graphicx}
\usepackage{grffile}
\usepackage{longtable}
\usepackage{wrapfig}
\usepackage{rotating}
\usepackage[normalem]{ulem}
\usepackage{amsmath}
\usepackage{textcomp}
\usepackage{amssymb}
\usepackage{capt-of}
\usepackage{hyperref}
\usepackage{color}
\usepackage{listings}
\usepackage[spanish]{babel}
\usecolortheme{rose}
\setbeamercolor{alerted text}{fg=Blue}
\setbeamerfont{alerted text}{series=\bfseries}
\setbeamercolor{block title}{bg=structure.fg!20!bg!50!bg}
\setbeamercolor{block body}{use=block title,bg=block title.bg}
\setbeamertemplate{navigation symbols}{}
\AtBeginSubsection[]{\begin{frame}[plain]\tableofcontents[currentsubsection,sectionstyle=show/shaded,subsectionstyle=show/shaded/hide]\end{frame}}
\lstset{keywordstyle=\color{blue}, commentstyle=\color{gray!90}, basicstyle=\ttfamily\small, columns=fullflexible, breaklines=true,linewidth=\textwidth, backgroundcolor=\color{gray!23}, basewidth={0.5em,0.4em}, literate={á}{{\'a}}1 {ñ}{{\~n}}1 {é}{{\'e}}1 {ó}{{\'o}}1 {º}{{\textordmasculine}}1, showstringspaces=false}
\usepackage{mathpazo}
\hypersetup{colorlinks=true, linkcolor=Blue, urlcolor=Blue}
\usepackage{fancyvrb}
\DefineVerbatimEnvironment{verbatim}{Verbatim}{fontsize=\tiny, formatcom = {\color{black!70}}}
\usetheme{Goettingen}
\usefonttheme{serif}
\author{Oscar Perpiñán Lamigueiro $\backslash$\ \url{http://oscarperpinan.github.io}}
\date{}
\title{Crear paquetes en R}
\hypersetup{
 pdfauthor={Oscar Perpiñán Lamigueiro $\backslash$\ \url{http://oscarperpinan.github.io}},
 pdftitle={Crear paquetes en R},
 pdfkeywords={},
 pdfsubject={},
 pdfcreator={Emacs 24.5.1 (Org mode 9.0.5)}, 
 pdflang={Spanish}}
\begin{document}

\maketitle


\section{Introducción}
\label{sec:org985fe1c}

\begin{frame}[label={sec:org268fdd0}]{Definiciones previas (\href{http://cran.r-project.org/doc/contrib/Leisch-CreatingPackages.pdf}{Creating R Packages: a tutorial})}
\begin{itemize}
\item \alert{Package}: An extension of the R base system with code, data and
documentation in standardized format.
\item \alert{Library}: A directory containing installed packages.
\item \alert{Repository}: A website providing packages for installation.
\end{itemize}
\end{frame}

\begin{frame}[label={sec:org938644a}]{Definiciones previas (\href{http://cran.r-project.org/doc/contrib/Leisch-CreatingPackages.pdf}{Creating R Packages: a tutorial})}
\begin{itemize}
\item \alert{Source}: The original version of a package with human-readable text and code.
\item \alert{Binary}: A compiled version of a package with computer-readable
text and code, may work only on a specific platform.
\end{itemize}
\end{frame}

\begin{frame}[label={sec:orgb052889}]{Definiciones previas (\href{http://cran.r-project.org/doc/contrib/Leisch-CreatingPackages.pdf}{Creating R Packages: a tutorial})}
\begin{itemize}
\item \alert{Base packages}: Part of the R source tree, maintained by R Core.
\item \alert{Recommended packages}: Part of every R installation, but not
necessarily maintained by R Core.
\item \alert{Contributed packages}: All the rest.
\end{itemize}
\end{frame}

\begin{frame}[fragile,label={sec:org4a4e147}]{¿Por qué crear y publicar un paquete para \texttt{R}?}
 \begin{itemize}
\item Es una herramienta cómoda para mantener colecciones coherentes de funciones
y datos.
\item Permite publicar código de forma que pueda ser empleado por
otros siguiendo unas estructuras comunes.
\item La estructura de un paquete obliga a organizar, limpiar y
documentar el código.
\item Al distribuir las herramientas para que otros puedan usarlas, se
obtiene realimentación sobre lo publicado, de forma que se
aumenta su robustez, se amplian sus funcionalidades y se conecta
con otras herramientas y proyectos.
\item \guillemotleft{}As we enjoy great advantages from the inventions of others, we
should be glad of an opportunity to serve others by any
invention of ours, and this we should do freely and generously.\guillemotright{}
\end{itemize}
\end{frame}

\begin{frame}[label={sec:org8045764}]{Algunos consejos genéricos}
Extraído de \href{http://arxiv.org/abs/1210.0530}{Best Practices for Scientific Computing}

\begin{itemize}
\item Write programs for people, not computers.
\item Automate repetitive tasks
\item Use the computer to record history
\item Make incremental changes
\item Use version control
\item Don't repeat yourself (or others)
\item Plan for mistakes
\item Optimize software only after it works correctly
\item Document design and purpose, not mechanics
\item Collaborate
\end{itemize}
\end{frame}

\begin{frame}[label={sec:org93af26b}]{}
\begin{block}{Referencias}
\begin{itemize}
\item \href{http://cran.r-project.org/doc/manuals/r-release/R-exts.html}{Writing R Extensions}
\item \href{http://cran.r-project.org/doc/contrib/Leisch-CreatingPackages.pdf}{Creating R Packages: a tutorial}
\item \href{http://kbroman.org/pkg\_primer/}{R package primer}
\item \href{http://portal.stats.ox.ac.uk/userdata/ruth/APTS2012/Rcourse10.pdf}{Making an R package}
\end{itemize}
\end{block}
\end{frame}

\section{Crear un paquete en R}
\label{sec:org7ff84e0}

\begin{frame}[fragile,label={sec:org2e375bc}]{Estructura}
 \begin{itemize}
\item Las fuentes de un paquete de \texttt{R} están contenidas en un
directorio que contiene al menos:
\begin{itemize}
\item Los ficheros DESCRIPTION y NAMESPACE
\item Los subdirectorios:
\begin{itemize}
\item \texttt{R}: código en ficheros .R
\item \texttt{man}: páginas de ayuda de las funciones, métodos y clases
contenidos en el paquete.
\end{itemize}
\end{itemize}
\item Esta estructura puede ser generada con \texttt{package.skeleton}.
\end{itemize}
\end{frame}

\begin{frame}[fragile,label={sec:org957e178}]{DESCRIPTION}
 \begin{itemize}
\item El fichero DESCRIPTION contiene la información básica sobre el
paquete con un formato preestablecido:
\end{itemize}
\begin{verbatim}
     Package: pkgname
     Version: 0.5-1
     Date: 2004-01-01
     Title: My First Collection of Functions
     Authors@R: c(person("Joe", "Developer", role = c("aut", "cre"),
                          email = "Joe.Developer@some.domain.net"),
                   person("Pat", "Developer", role = "aut"),
                   person("A.", "User", role = "ctb",
     	             email = "A.User@whereever.net"))
     Author: Joe Developer and Pat Developer, with contributions from A. User
     Maintainer: Joe Developer <Joe.Developer@some.domain.net>
     Depends: R (>= 1.8.0), nlme
     Suggests: MASS
     Description: A short (one paragraph) description of what
       the package does and why it may be useful.
     License: GPL (>= 2)
     URL: http://www.r-project.org, http://www.another.url
\end{verbatim}
\begin{itemize}
\item Los campos \texttt{Package}, \texttt{Version}, \texttt{License}, \texttt{Title}, \texttt{Author} y
\texttt{Maintainer} son obligatorios.
\end{itemize}
\end{frame}
\begin{frame}[fragile,label={sec:org7aef7b1}]{NAMESPACE}
 \begin{itemize}
\item \texttt{R} usa un sistema de gestión de \emph{espacio de nombres} que
permite al autor del paquete especificar:
\begin{itemize}
\item las variables del paquete que se exportan (y son, por tanto,
accesibles a los usuarios)
\item las variables que se importan de otros paquetes.
\item las clases y métodos \texttt{S3} y \texttt{S4} que deben registrarse.
\end{itemize}
\item Este mecanismo queda definido en el contenido del fichero
\texttt{NAMESPACE}.
\end{itemize}
\end{frame}
\begin{frame}[fragile,label={sec:org9b46084}]{NAMESPACE}
 El \texttt{NAMESPACE} controla la estrategia de búsqueda de variables
  que utilizan las funciones del paquete:
\begin{itemize}
\item En primer lugar busca entre las creadas localmente (por el código de la carpeta \texttt{R/}).
\item En segundo lugar busca entre las variables importadas
explícitamente de otros paquetes.
\item En tercer lugar busca en el \texttt{NAMESPACE} del paquete \texttt{base}.
\item Por último busca siguiendo el camino habitual (ver el
resultado de \texttt{search()})
\end{itemize}
\end{frame}

\begin{frame}[fragile,label={sec:orgc1e5574}]{NAMESPACE}
 \begin{itemize}
\item Para exportar las variables \texttt{f} y \texttt{g}:
\end{itemize}
\lstset{language=r,label= ,caption= ,captionpos=b,numbers=none}
\begin{lstlisting}
export(f, g)
\end{lstlisting}
\begin{itemize}
\item Para importar \alert{todas} las variables del paquete \texttt{pkgExt}:
\end{itemize}
\lstset{language=r,label= ,caption= ,captionpos=b,numbers=none}
\begin{lstlisting}
import(pkgExt)
\end{lstlisting}
\begin{itemize}
\item Para importar las variables \texttt{var1} y \texttt{var2} del paquete
\texttt{pkgExt}:
\end{itemize}
\lstset{language=r,label= ,caption= ,captionpos=b,numbers=none}
\begin{lstlisting}
importFrom(pkgExt, var1, var2)
\end{lstlisting}
\end{frame}

\begin{frame}[fragile,label={sec:org0e8d0db}]{NAMESPACE}
 \begin{itemize}
\item Para registrar el método \texttt{S3} \texttt{print} para la clase \texttt{myClass}:
\end{itemize}
\lstset{language=r,label= ,caption= ,captionpos=b,numbers=none}
\begin{lstlisting}
S3method(print, myClass)
\end{lstlisting}
\begin{itemize}
\item Para registrar las clases \texttt{S4} \texttt{class1} y \texttt{class2}:
\end{itemize}
\lstset{language=r,label= ,caption= ,captionpos=b,numbers=none}
\begin{lstlisting}
exportClasses(class1, class2)
\end{lstlisting}
\begin{itemize}
\item Para registrar los métodos \texttt{S4} \texttt{method1} y \texttt{method2}:
\end{itemize}
\lstset{language=r,label= ,caption= ,captionpos=b,numbers=none}
\begin{lstlisting}
exportMethods(method1, method2)
\end{lstlisting}
\begin{itemize}
\item Para importar métodos y clases \texttt{S4} de otro paquete:
\end{itemize}
\lstset{language=r,label= ,caption= ,captionpos=b,numbers=none}
\begin{lstlisting}
importClassesFrom(package, ...)
importMethodsFrom(package, ...)
\end{lstlisting}
\end{frame}

\begin{frame}[fragile,label={sec:orgefccd04}]{Documentación}
 \begin{itemize}
\item Las páginas de ayuda de los objetos \texttt{R} se escriben usando el
formato “R documentation” (Rd), un lenguaje similar a \LaTeX{}.
\item Es aconsejable seguir estas orientaciones: \href{http://developer.r-project.org/Rds.html}{Guidelines for Rd files}
\item Para generar el esqueleto de un fichero Rd es aconsejable usar:
\begin{itemize}
\item \texttt{prompt}: \href{http://cran.r-project.org/doc/manuals/r-release/R-exts.html\\\#Documenting-functions}{genérica}
\item \texttt{promptClass} y \texttt{promptMethods}: \href{http://cran.r-project.org/doc/manuals/r-release/R-exts.html\\\#Documenting-S4-classes-and-methods}{clases y métodos}.
\item \texttt{promptPackage}: \href{http://cran.r-project.org/doc/manuals/r-release/R-exts.html\\\#Documenting-packages}{paquete}
\item \texttt{promptData}: \href{http://cran.r-project.org/doc/manuals/r-release/R-exts.html\\\#Documenting-data-sets}{datos}
\end{itemize}
\item Todos los comandos disponibles están en el documento \href{http://developer.r-project.org/parseRd.pdf}{Parsing Rd files}.
\end{itemize}
\end{frame}

\begin{frame}[fragile,label={sec:org5c47959}]{}
 \begin{verbatim}
  \name{load}
  \alias{load}
  \title{Reload Saved Datasets}
  \description{
    Reload the datasets written to a file with the function
    \code{save}.
  }
  \usage{
    load(file, envir = parent.frame())
  }
  \arguments{
  \item{file}{a connection or a character string giving the
      name of the file to load.}
  \item{envir}{the environment where the data should be
      loaded.}
  }
  \seealso{
    \code{\link{save}}.
  }
  \examples{
    ## save all data
    save(list = ls(), file= "all.RData")
    
    ## restore the saved values to the current environment
    load("all.RData")
    
    ## restore the saved values to the workspace
    load("all.RData", .GlobalEnv)
  }
  \keyword{file}
\end{verbatim}
\end{frame}

\section{Publicar un paquete}
\label{sec:org11ef8f3}

\begin{frame}[fragile,label={sec:orgad905ca}]{Itinerario}
 \begin{itemize}
\item Comprobar
\end{itemize}
\begin{verbatim}
R CMD check myPackage/
\end{verbatim}
\begin{itemize}
\item Construir
\end{itemize}
\begin{verbatim}
R CMD build myPackage/
\end{verbatim}
\begin{itemize}
\item Publicar (o actualizar) en un repositorio
\end{itemize}
\end{frame}

\begin{frame}[fragile,label={sec:org10f0da4}]{Comprobar}
 \begin{itemize}
\item Comprobar un directorio (desde línea de comandos):
\end{itemize}
\begin{verbatim}
R CMD check myPackage/
\end{verbatim}
\begin{itemize}
\item Comprobar un paquete ya construido (desde línea de comandos):
\end{itemize}
\begin{verbatim}
R CMD check myPackage.tar.gz
\end{verbatim}
\begin{itemize}
\item Esta comprobación incluye más de 20 puntos de prueba detallados
en el manual \href{http://cran.r-project.org/doc/manuals/R-exts.html\#Checking-packages}{Writing R extensions}.
\end{itemize}
\end{frame}
\begin{frame}[fragile,label={sec:org229718c}]{Construir}
 \begin{block}{Fuente o binario}
Se puede construir un fichero fuente en formato \emph{tarball}
(independiente de la plataforma, habitual en sistemas Unix) o en
forma binaria (dependiente de la plataforma, habitual para Windows y Mac).
\end{block}
\begin{itemize}
\item Fuente en formato \emph{tarball}: el resultado es un fichero \emph{tarball} \texttt{myPackage.tar.gz} que
se puede distribuir a cualquier sistema.
\end{itemize}
\begin{verbatim}
R CMD build myPackage/
\end{verbatim}
\begin{itemize}
\item Comprimido binario: el resultado es una copia comprimida de la versión
\alert{instalada} del paquete: depende del sistema operativo.
\end{itemize}
\begin{verbatim}
R CMD INSTALL --build myPackage/
\end{verbatim}
\end{frame}

\begin{frame}[fragile,label={sec:org7257dd9}]{Comprobar y construir en sistemas Windows}
 \begin{itemize}
\item Para paquetes sin código compilado (C, Fortran), también se puede usar
\texttt{R CMD check} y \texttt{R CMD build} en un sistema Windows.
\item Para generar un binario hay que usar \texttt{R CMD INSTALL -{}-build}.
\begin{itemize}
\item Es posible que haya que modificar la variables de entorno
\texttt{TEMP} y \texttt{TMP} de forma que \alert{sólo} contengan caracteres ASCII.
\end{itemize}
\item Para paquetes con código compilado, o en caso de problemas con
los comandos anteriores, hay que usar \href{http://cran.r-project.org/bin/windows/Rtools/}{Rtools}.
\item Se pueden instalar fuentes \emph{tarball} con (ver \href{http://cran.r-project.org/doc/manuals/R-admin.html\#Windows-packages}{R installation and administration}):
\end{itemize}
\begin{verbatim}
install.packages(myPackage.tar.gz, type='source')
\end{verbatim}
\end{frame}


\begin{frame}[fragile,label={sec:orgcb384dd}]{Repositorios: CRAN}
 El principal repositorio de paquetes estables es \href{http://cran.r-project.org/}{CRAN}.
\begin{itemize}
\item Publicar en este repositorio conlleva la aceptación de unas \href{http://cran.r-project.org/web/packages/policies.html}{condiciones}.
\item Para publicar en CRAN hay que subir el fichero fuente \emph{tarball}
resultado de \texttt{R CMD build} mediante el formulario web
\url{https://cran.r-project.org/submit.html} siguiendo los pasos que
allí se indican.
\item Es imprescindible haber comprobado el fichero con \texttt{R CMD check
    -{}-as-cran} antes de subirlo al formulario. El resultado de
esta comprobación \alert{no} debe contener errores, advertencias o notas.
\item Más detalle en el apartado \emph{Submission} de las \href{http://cran.r-project.org/web/packages/policies.html}{condiciones de
CRAN} y en el manual \href{http://cran.r-project.org/doc/manuals/R-exts.html\\\#Submitting-a-package-to-CRAN}{R Extensions}.
\end{itemize}
\end{frame}

\begin{frame}[label={sec:orgc367e01}]{Repositorios}
Otros repositorios destacables son:

\begin{itemize}
\item \href{https://github.com/}{GitHub} (versiones de desarrollo)

\item \href{http://r-forge.r-project.org/}{R-Forge} (versiones de desarrollo)

\item \href{http://rforge.net/}{RForge} (versiones de desarrollo)

\item \href{http://www.bioconductor.org/}{Bioconductor} (paquetes de bioinformática)
\end{itemize}
\end{frame}

\begin{frame}[fragile,label={sec:org75b9557}]{Repositorios: GitHub}
 \begin{itemize}
\item GitHub es actualmente el repositorio de código más importante por número de usuarios y proyectos alojados.
\begin{itemize}
\item Documentación: \href{https://guides.github.com/}{Guías sobre GitHub}.
\item Existe una \href{https://desktop.github.com/}{aplicación de escritorio} (\href{https://help.github.com/desktop/guides/getting-started/}{guía introductoria})
\end{itemize}
\item Emplea \texttt{git} para realizar control de versiones.
\begin{itemize}
\item \href{https://help.github.com/articles/good-resources-for-learning-git-and-github/}{Recursos} para aprender a usar \texttt{git}.
\end{itemize}
\item \href{https://github.com/search?o=desc\&q=language:R\&ref=searchresults\&s=stars\&type=Repositories}{Proyectos de \texttt{R} en GitHub}.
\end{itemize}
\end{frame}
\end{document}