% Created 2013-02-21 jue 11:41
\documentclass[xcolor={usenames,svgnames,dvipsnames}]{beamer}
\usepackage[utf8]{inputenc}
\usepackage[T1]{fontenc}
\usepackage{fixltx2e}
\usepackage{graphicx}
\usepackage{longtable}
\usepackage{float}
\usepackage{wrapfig}
\usepackage{soul}
\usepackage{textcomp}
\usepackage{marvosym}
\usepackage{wasysym}
\usepackage{latexsym}
\usepackage{amssymb}
\usepackage{hyperref}
\tolerance=1000
\usepackage{color}
\usepackage{listings}
\AtBeginSection[]{\begin{frame}<beamer>\frametitle{Contenidos}\tableofcontents[currentsection]\end{frame}}
\lstset{keywordstyle=\color{blue}, commentstyle=\color{gray!90}, basicstyle=\ttfamily\small, columns=fullflexible, breaklines=true,linewidth=\textwidth, backgroundcolor=\color{gray!23}, basewidth={0.5em,0.4em}, literate={á}{{\'a}}1 {ñ}{{\~n}}1 {é}{{\'e}}1 {ó}{{\'o}}1 {º}{{\textordmasculine}}1}
\usepackage{mathpazo}
\setbeamercovered{transparent}
\usefonttheme{serif} 
\usetheme{Goettingen}
\hypersetup{colorlinks=true, linkcolor=Blue, urlcolor=Blue}
\usepackage{fancyvrb}
\DefineVerbatimEnvironment{verbatim}{Verbatim}{fontsize=\tiny, formatcom = {\color{black!70}}}
\providecommand{\alert}[1]{\textbf{#1}}

\title{Manejo de datos con R}
\author{Oscar Perpiñán Lamigueiro}
\date{25 de Enero de 2013}
\hypersetup{
  pdfkeywords={},
  pdfsubject={},
  pdfcreator={Emacs Org-mode version 7.8.11}}

\begin{document}

\maketitle


\section{Fuentes de datos}
\label{sec-1}
\begin{frame}
\frametitle{Fuentes de datos}
\label{sec-1-1}


\begin{itemize}
\item \href{http://stat.ethz.ch/R-manual/R-patched/library/datasets/html/00Index.html}{The R Datasets Package}
\item \href{http://www.bibsonomy.org/user/procomun/data?resourcetype=bookmark}{Enlaces en Bibsonomy}
\end{itemize}
\end{frame}
\section{Lectura de datos}
\label{sec-2}
\begin{frame}[fragile]
\frametitle{\texttt{setwd}, \texttt{getwd}, \texttt{dir}}
\label{sec-2-1}


\lstset{language=R}
\begin{lstlisting}
getwd()
old <- setwd("~/R/intro")
dir()
dir(pattern='.R')
dir('data')
\end{lstlisting}
\end{frame}
\begin{frame}[fragile]
\frametitle{read.table}
\label{sec-2-2}

\begin{itemize}
\item Con un fichero local
\end{itemize}

\lstset{language=R}
\begin{lstlisting}
download.file('http://oscarperpinan.github.com/spacetime-vis/data/CO2_GNI_BM.csv',
              destfile='data/CO2_GNI_BM.csv')

CO2 <- read.table('data/CO2_GNI_BM.csv', header=TRUE, sep=',')
head(CO2)
\end{lstlisting}
\begin{itemize}
\item Directamente de un enlace URL
\end{itemize}

\lstset{language=R}
\begin{lstlisting}
CO2 <- read.table('http://oscarperpinan.github.com/spacetime-vis/data/CO2_GNI_BM.csv',
                header=TRUE, sep=',')
head(CO2)
\end{lstlisting}
\end{frame}
\begin{frame}[fragile]
\frametitle{read.csv, read.csv2}
\label{sec-2-3}

\begin{itemize}
\item \texttt{read.csv} y \texttt{read.csv2} son como \texttt{read.table} con valores
  por defecto para encabezado y separadores
\end{itemize}


\lstset{language=R}
\begin{lstlisting}
CO2 <- read.csv('data/CO2_GNI_BM.csv')
\end{lstlisting}


\lstset{language=R}
\begin{lstlisting}
head(CO2)
tail(CO2)
summary(CO2)
names(CO2)
\end{lstlisting}
\end{frame}
\section{Datos agregados}
\label{sec-3}
\begin{frame}[fragile]
\frametitle{table}
\label{sec-3-1}


\lstset{language=R}
\begin{lstlisting}
chromo <- data.frame(chromosome = gl(3,  10,
                     labels = c('A',  'B',  'C')),
                     probeset = gl(3,  10,
                     labels = c('X',  'Y',  'Z')),
                     ensg =  gl(3,  10,
                     labels = c('E1',  'E2',  'E3')),
                     symbol = gl(3,  10,
                     labels = c('S1',  'S2',  'S3')),
                     XXA_00 = rnorm(30),
                     XXA_36 = rnorm(30),
                     XXB_00 = rnorm(30))
\end{lstlisting}


\lstset{language=R}
\begin{lstlisting}
table(chromo$chromosome, df$XXA_00 > 0)
table(chromo$probeset, df$XXA_00 > -1 & df$XXA_00 < 1)
\end{lstlisting}
\end{frame}
\begin{frame}[fragile]
\frametitle{xtabs}
\label{sec-3-2}


\lstset{language=R}
\begin{lstlisting}
xtabs(XXA_00 > 1 ~ chromosome + probeset,
      data=chromo)
\end{lstlisting}
\end{frame}
\begin{frame}[fragile]
\frametitle{tapply}
\label{sec-3-3}


\lstset{language=R}
\begin{lstlisting}
tapply(CO2$X2000, CO2$Indicator.Name,
       FUN=mean)
\end{lstlisting}

\begin{verbatim}
           CO2 emissions (kg per PPP $ of GDP) 
                                  4.777875e-01 
        CO2 emissions (metric tons per capita) 
                                  7.580861e+00 
 GNI per capita, PPP (current international $) 
                                  1.981000e+04 
            GNI, PPP (current international $) 
                                  2.078196e+12
\end{verbatim}
\end{frame}
\begin{frame}[fragile]
\frametitle{tapply}
\label{sec-3-4}


\lstset{language=R}
\begin{lstlisting}
tapply(CO2$X2000, CO2[,c("Indicator.Name", "Country.Name")],
       FUN=mean)
\end{lstlisting}


\begin{verbatim}
                                               Country.Name
Indicator.Name                                        Brazil        China
  CO2 emissions (kg per PPP $ of GDP)           2.699746e-01 1.140619e+00
  CO2 emissions (metric tons per capita)        1.892645e+00 2.696862e+00
  GNI per capita, PPP (current international $) 6.820000e+03 2.340000e+03
  GNI, PPP (current international $)            1.188790e+12 2.948850e+12
                                               Country.Name
Indicator.Name                                       Finland       France
  CO2 emissions (kg per PPP $ of GDP)           3.923481e-01 2.384221e-01
  CO2 emissions (metric tons per capita)        1.007322e+01 6.016236e+00
  GNI per capita, PPP (current international $) 2.548000e+04 2.566000e+04
  GNI, PPP (current international $)            1.318800e+11 1.558990e+12
                                               Country.Name
Indicator.Name                                       Germany       Greece
  CO2 emissions (kg per PPP $ of GDP)           3.929031e-01 4.598579e-01
  CO2 emissions (metric tons per capita)        1.012147e+01 8.391709e+00
  GNI per capita, PPP (current international $) 2.549000e+04 1.832000e+04
  GNI, PPP (current international $)            2.095450e+12 2.000130e+11
                                               Country.Name
Indicator.Name                                         India       Norway
  CO2 emissions (kg per PPP $ of GDP)           7.448517e-01 2.391275e-01
  CO2 emissions (metric tons per capita)        1.125975e+00 8.641315e+00
  GNI per capita, PPP (current international $) 1.500000e+03 3.565000e+04
  GNI, PPP (current international $)            1.575930e+12 1.601000e+11
                                               Country.Name
Indicator.Name                                         Spain United States
  CO2 emissions (kg per PPP $ of GDP)           3.428950e-01  5.568755e-01
  CO2 emissions (metric tons per capita)        7.312922e+00  1.953626e+01
  GNI per capita, PPP (current international $) 2.115000e+04  3.569000e+04
  GNI, PPP (current international $)            8.514620e+11  1.007050e+13
\end{verbatim}
\end{frame}
\begin{frame}[fragile]
\frametitle{aggregate}
\label{sec-3-5}


\lstset{language=R}
\begin{lstlisting}
aggregate(X2000 ~ Indicator.Name,
          data=CO2, FUN=mean)

aggregate(cbind(X2000, X2001) ~ Indicator.Name,
          data=CO2, FUN=mean)

aggregate(X2000 ~ Indicator.Name + Country.Name,
          data=CO2, FUN=mean)
\end{lstlisting}
\end{frame}
\begin{frame}[fragile]
\frametitle{aggregate}
\label{sec-3-6}


\lstset{language=R}
\begin{lstlisting}
aggregate(cbind(X2000, X2001) ~
          Indicator.Name + Country.Name,
          data=CO2, FUN=mean)

aggregate(cbind(X2000, X2001) ~
          Indicator.Name + Country.Name,
          data=CO2, FUN=mean)

aggregate(cbind(X2000, X2001) ~
          Indicator.Name + Country.Name,
          subset=(Country.Name %in% c('United States', 'China')),
                  data=CO2, FUN=mean)
\end{lstlisting}
\end{frame}
\begin{frame}[fragile]
\frametitle{aggregate}
\label{sec-3-7}


\lstset{language=R}
\begin{lstlisting}
aggregate(cbind(XXA_00, XXA_36, XXB_00) ~
          ensg + chromosome + symbol,
          data = chromo,  FUN = mean)

aggregate(cbind(XXA_00, XXA_36, XXB_00) ~ ensg ,
          data = chromo,  FUN = mean)
\end{lstlisting}
\end{frame}
\section{Cambio de formato}
\label{sec-4}
\begin{frame}[fragile]
\frametitle{\texttt{stack}}
\label{sec-4-1}

\begin{itemize}
\item Primero escogemos un subconjunto
\end{itemize}

\lstset{language=R}
\begin{lstlisting}
CO2China <- subset(CO2,
                   subset=(Country.Name=='China' &
                           Indicator.Name=='CO2 emissions (kg per PPP $ of GDP)'),
                   select=-c(Country.Name, Country.Code,
                             Indicator.Name, Indicator.Code))
\end{lstlisting}
\begin{itemize}
\item Pasamos de formato \texttt{wide} a \texttt{long}
\end{itemize}

\lstset{language=R}
\begin{lstlisting}
stack(CO2China)
\end{lstlisting}
\end{frame}
\begin{frame}[fragile]
\frametitle{\texttt{reshape}: \texttt{wide} a \texttt{long}}
\label{sec-4-2}

\begin{itemize}
\item Primer intento
\end{itemize}

\lstset{language=R}
\begin{lstlisting}
CO2long <- reshape(CO2,
                   varying=list(names(CO2)[5:16]),
                   direction='long')
head(CO2long)
\end{lstlisting}
\begin{itemize}
\item Añadimos argumentos
\end{itemize}

\lstset{language=R}
\begin{lstlisting}
CO2long <- reshape(CO2,
                   varying=list(names(CO2)[5:16]),
                   timevar='Year', v.names='Value',
                   times=2000:2011,
                   direction='long')
head(CO2long)
\end{lstlisting}
\end{frame}
\begin{frame}[fragile]
\frametitle{\texttt{reshape}: \texttt{long} a \texttt{wide}}
\label{sec-4-3}

\begin{itemize}
\item Primero escogemos las columnas de interés
\end{itemize}

\lstset{language=R}
\begin{lstlisting}
CO2subset <- CO2long[c("Country.Name",
                       "Indicator.Name",
                       "Year", "Value")]
\end{lstlisting}
\begin{itemize}
\item Ahora cambiamos formato
\end{itemize}

\lstset{language=R}
\begin{lstlisting}
CO2wide <- reshape(CO2subset,
                   idvar=c('Country.Name','Year'),
                   timevar='Indicator.Name',
                   direction='wide')
\end{lstlisting}
\begin{itemize}
\item Y ponemos nombres al gusto
\end{itemize}

\lstset{language=R}
\begin{lstlisting}
names(CO2wide)[3:6] <- c('CO2.PPP', 'CO2.capita',
                         'GNI.PPP', 'GNI.capita')

head(CO2wide)
\end{lstlisting}

  
  
  
  
  
  
  
\end{frame}

\end{document}